%versi 2 (8-10-2016) 
\chapter{Pendahuluan}
\label{chap:intro}
   
\section{Latar Belakang}
\label{sec:label}

Setiap dosen wali (dikenal juga dengan istilah penasehat akademik \footnote{Sebagaimana dituliskan pada \url{https://unpar.ac.id/akademik/}. Pada skripsi ini istilah yang digunakan adalah dosen wali karena istilah tersebut yang muncul pada aplikasi SIAKAD \cite{siakad}}) memiliki data mengenai mahasiswa walinya yang dapat diakses melalui SIAKAD \cite{siakad}. Namun, walaupun dosen wali memiliki data mengenai mahasiswa walinya, dosen wali juga perlu melakukan pemeriksaan data mahasiswa walinya, terutama data akademiknya secara berkala. Dengan berbagai kesibukan yang dialami oleh para dosen wali dan mahasiswa, ditambah dengan situasi Indonesia saat ini yang menyebabkan perkuliahan dilakukan secara daring, akan sangat sulit bagi dosen wali untuk menemui mahasiswa walinya. Hal ini menyebabkan dosen wali kesulitan mengamati perkembangan mahasiswa walinya. Selain itu dalam mencari data akademik mahasiswa wali, SIAKAD juga memiliki kekurangan dimana dosen wali perlu melakukan \textit{login}, kemudian mencari serta memilih mahasiswa wali yang ingin dilihat. Sehingga dapat dikatakan proses tersebut tidaklah efisien.

Maka dari itu, pada skripsi ini akan dibuat sebuah perangkat lunak yang berupa \textit{screensaver} yang dapat menampilkan data akademik mahasiswa wali secara acak. \textit{Screensaver} adalah program komputer yang mengosongkan layar atau mengisinya dengan gambar atau pola bergerak ketika komputer telah diam dalam waktu yang lama \cite{screensaver}. Dengan menggunakan perangkat lunak tersebut, dosen wali dapat tetap mengamati perkembangan mahasiswa walinya, paling tidak secara akademik.

Dikarenakan terbimbing tidak memiliki akses ke SIAKAD untuk mengakses data mahasiswa wali, namun terbimbing memiliki akses ke Portal Akademik Mahasiswa \cite{stupor} maka, terbimbing mensimulasikan dengan Portal Akademik Mahasiswa, dan kemudian Pembimbing mengubah aksesnya ke SIAKAD. Struktur kelas yang akan digunakan dalam pembuatan perangkat lunak ini yaitu, struktur kelas SIAModels yang tersedia pada Github dan Maven Public Repository \cite{siamodels}. Simulasi Portal Akademik Mahasiswa ini berdasarkan pada skripsi yang dibuat oleh Andrianto Sugiarto \cite{ifstupor}. Tetapi terdapat beberapa perbaikan yang perlu dilakukan, dan dapat dilihat pada sub-bab \ref{analisisPemanfaatanJsoup}.

Perangkat lunak akan dibangun menggunakan bahasa pemrograman Java. Terdapat beberapa teknologi yang dapat dimanfaatkan dalam bahasa pemrograman Java. Teknologi yang pertama yaitu \textit{library} jsoup. Jsoup dapat digunakan untuk melakukan \textit{scraping}, sehingga pengambilan data mahasiswa tidak memerlukan API \textit{(Application Programming Interface)} \cite{jsoup}. Teknologi lainnya yang dapat dimanfaatkan yaitu JavaFX. JavaFX dapat digunakan untuk mengonversi aplikasi tersebut menjadi \textit{screensaver}.



\section{Rumusan Masalah}
\label{sec:rumusan}
Rumusan masalah yang akan dibahas pada skripsi ini adalah sebagai berikut:
\begin{itemize}
	\item Bagaimana cara memanfaatkan jsoup untuk mengambil data mahasiswa?
	\item Bagaimana cara memanfaatkan JavaFX untuk mengonversi aplikasi tersebut menjadi \textit{screensaver}?
\end{itemize}   

\section{Tujuan}
\label{sec:tujuan}
Tujuan yang ingin dicapai dari penulisan skripsi ini sebagai berikut:
\begin{itemize}
    \item Memanfaatkan jsoup untuk mengambil data mahasiswa.
    \item Memanfaatkan JavaFX untuk mengonversi aplikasi tersebut menjadi \textit{screensaver}.
\end{itemize}

\section{Batasan Masalah}
\label{sec:batasan}
Dikarenakan terbimbing tidak memiliki akses ke SIAKAD untuk mengakses data mahasiswa wali, namun terbimbing memiliki akses ke Portal Akademik Mahasiswa maka, terbimbing mensimulasikan dengan Portal Akademik Mahasiswa, dan kemudian Pembimbing mengubah aksesnya ke SIAKAD. 

\section{Metodologi}
\label{sec:metlit}
Langkah-langkah yang akan dilakukan dalam melakukan penelitian ini yaitu:
	\begin{enumerate}
		\item Melakukan studi mengenai jsoup.
		\item Melakukan studi mengenai cara mengonversi aplikasi menjadi \textit{screensaver}.
		\item Mempelajari struktur kelas SIAModels.
		\item Menganalisis IF Portal Akademik Mahasiswa dan Portal Akademik Mahasiswa.
		\item Merancang struktur kelas aplikasi.
		\item Melakukan studi mengenai cara mendesain antarmuka aplikasi
	    \item Mendesain antarmuka aplikasi.
	    \item Mengimplementasikan jsoup untuk mengambil data mahasiswa.
        \item Mengonversi aplikasi menjadi \textit{screensaver} dengan menggunakan JavaFX. 
		\item Melakukan pengujian dan eksperimen.
		\item Menulis dokumen skripsi.
	\end{enumerate}

\section{Sistematika Pembahasan}
\label{sec:sispem}
Dokumen dibagi ke dalam beberapa bab dengan sistematika pembahasan sebagai berikut:
    \begin{itemize}
        \item Bab 1. Pendahuluan, membahas tentang latar belakang, rumusan masalah, tujuan penelitian, batasan masalah, metode penelitian dan sistematika pembahasan mengenai skripsi.
        \item Bab 2. Landasan Teori, membahas landasan dari teori-teori yang berhubungan serta mendukung penelitian, meliputi jsoup, JavaFX, dan SIAModels. 
        \item Bab 3. Analisis, menjelaskan tentang analisis Portal Akademik Mahasiswa, analisis SIAKAD, analisis data yang dibutuhkan untuk \textit{screensaver}, serta analisis sistem \textit{screensaver}.
        \item Bab 4. Perancangan, membahas perancangan kelas beserta deskripsi kelas dan fungsinya, serta perancangan antarmuka.
        \item Bab 5. Implementasi dan pengujian, membahas hasil-hasil implementasi dan pengujian secara fungsional dan eksperimental.
        \item Bab 6. Kesimpulan dan saran, membahas kesimpulan yang diperoleh dari penelitian ini dan saran untuk pengembangan berikutnya.
    \end{itemize}