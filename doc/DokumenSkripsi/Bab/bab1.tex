%versi 2 (8-10-2016) 
\chapter{Pendahuluan}
\label{chap:intro}
   
\section{Latar Belakang}
\label{sec:label}

Setiap Penasehat Akademik memiliki data mengenai mahasiswa walinya. Namun, walaupun Penasehat Akademik memiliki data mengenai mahasiswa walinya, Penasehat Akademik juga perlu melakukan pemeriksaan data mahasiswa walinya, terutama data akademiknya secara berkala. Dengan berbagai kesibukan yang dialami oleh para Penasehat Akademik dan mahasiswa, ditambah dengan situasi Indonesia saat ini yang menyebabkan perkuliahan dilakukan secara daring, akan sangat sulit bagi Penasehat Akademik untuk menemui mahasiswa walinya. Hal ini menyebabkan Penasehat Akademik kesulitan mengamati perkembangan mahasiswa walinya. 

Maka dari itu, pada skripsi ini akan dibuat sebuah perangkat lunak yang berupa \textit{screensaver} yang dapat menampilkan data akademik mahasiswa wali secara acak. \textit{A screensaver (or screen saver) is a computer program that blanks the screen or fills it with moving images or patterns when the computer has been idle for a long time \cite{screensaver}.} Dengan menggunakan perangkat lunak tersebut, Penasehat Akademik dapat tetap mengamati perkembangan mahasiswa walinya, paling tidak secara akademik.

Dikarenakan terbimbing tidak memiliki akses ke SIAKAD \cite{siakad} untuk mengakses data mahasiswa wali, namun terbimbing memiliki akses ke Portal Akademik Mahasiswa \cite{stupor} maka, terbimbing mensimulasikan dengan Portal Akademik Mahasiswa, dan kemudian Pembimbing mengubah aksesnya ke SIAKAD. Struktur kelas yang akan digunakan dalam pembuatan perangkat lunak ini yaitu, struktur kelas SIAModels yang tersedia pada Github dan Maven Public Repository \cite{siamodels}.

Perangkat lunak akan dibangun menggunakan bahasa pemrograman Java. Terdapat beberapa teknologi yang dapat dimanfaatkan dalam bahasa pemrograman Java. Teknologi yang pertama yaitu \textit{library} jsoup. Jsoup dapat digunakan untuk melakukan \textit{web scraping}, sehingga pengambilan data mahasiswa tidak memerlukan API \textit{(Application Programming Interface)} \cite{jsoup}. Teknologi lainnya yang dapat dimanfaatkan yaitu JavaFX. JavaFX dapat digunakan untuk mengonversi aplikasi tersebut menjadi \textit{screensaver}.



\section{Rumusan Masalah}
\label{sec:rumusan}
Rumusan masalah yang akan dibahas pada skripsi ini adalah sebagai berikut:
\begin{itemize}
	\item Bagaimana cara memanfaatkan jsoup untuk mengambil data mahasiswa?
	\item Bagaimana cara memanfaatkan JavaFX untuk mengonversi aplikasi tersebut menjadi \textit{screensaver}?
\end{itemize}   

\section{Tujuan}
\label{sec:tujuan}
Tujuan yang ingin dicapai dari penulisan skripsi ini sebagai berikut:
\begin{itemize}
    \item Memanfaatkan jsoup untuk mengambil data mahasiswa.
    \item Memanfaatkan JavaFX untuk mengonversi aplikasi tersebut menjadi \textit{screensaver}.
\end{itemize}

\section{Batasan Masalah}
\label{sec:batasan}
Dikarenakan terbimbing tidak memiliki akses ke SIAKAD untuk mengakses data mahasiswa wali, namun terbimbing memiliki akses ke Portal Akademik Mahasiswa maka, terbimbing mensimulasikan dengan Portal Akademik Mahasiswa, dan kemudian Pembimbing mengubah aksesnya ke SIAKAD. 

\section{Metodologi}
\label{sec:metlit}
Langkah-langkah yang akan dilakukan dalam melakukan penelitian ini yaitu:
	\begin{enumerate}
		\item Melakukan studi mengenai jsoup.
		\item Melakukan studi mengenai cara mengonversi aplikasi menjadi \textit{screensaver}.
		\item Mempelajari struktur kelas SIAModels.
		\item Menganalisis IF Portal Akademik Mahasiswa dan Portal Akademik Mahasiswa.
		\item Merancang struktur kelas aplikasi.
		\item Melakukan studi mengenai cara mendesain antarmuka aplikasi
	    \item Mendesain antarmuka aplikasi.
	    \item Mengimplementasikan jsoup untuk mengambil data mahasiswa.
        \item Mengonversi aplikasi menjadi \textit{screensaver} dengan menggunakan JavaFX. 
		\item Melakukan pengujian dan eksperimen.
		\item Menulis dokumen skripsi.
	\end{enumerate}

\section{Sistematika Pembahasan}
\label{sec:sispem}
Dokumen dibagi ke dalam beberapa bab dengan sistematika pembahasan sebagai berikut:
    \begin{itemize}
        \item Bab 1. Pendahuluan, membahas tentang latar belakang, rumusan masalah, tujuan penelitian, batasan masalah, metode penelitian dan sistematika pembahasan mengenai skripsi.
        \item Bab 2. Landasan Teori, membahas landasan dari teori-teori yang berhubungan serta mendukung penelitian, meliputi \textit{web scraping}, jsoup, JavaFX, dan SIAModels. 
        \item Bab 3. Analisis, menjelaskan tentang kebutuhan data, analisis Portal Akademik Mahasiswa versi 2020, dan analisis cara memanfaatkan jsoup.
        \item Bab 4. Perancangan, membahas perancangan antarmuka, diagram kelas beserta deskripsi kelas dan fungsinya.
        \item Bab 5. Implementasi dan pengujian, membahas hasil-hasil implementasi dan pengujian secara fungsional dan eksperimental.
        \item Bab 6. Kesimpulan dan saran, membahas kesimpulan yang diperoleh dari penelitian ini dan saran untuk pengembangan berikutnya.
    \end{itemize}