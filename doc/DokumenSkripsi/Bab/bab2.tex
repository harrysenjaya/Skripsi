%versi 3 (22-07-2020)
\chapter{Landasan Teori}
\label{chap:teori}

Bab Landasan Teori ini berisi teori-teori yang menjadi dasar penelitian ini, meliputi JSOUP, dan Java Swing.

\section{JSOUP}
\label{sec:JSOUP} 
 
Menurut Bo Zhao\cite{zhao}, \textit{web scraping} juga dikenal sebagai ekstraksi atau pemanenan web, adalah teknik untuk mengekstrak data dari \textit{World Wide Web} (WWW) dan menyimpannya ke sistem file atau basis data untuk diambil atau dianalisis. Salah satu teknologi yang dapat dimanfaatkan untuk melakukan \textit{web scraping} yaitu \textit{library} Java JSOUP. JSOUP adalah \textit{library} Java untuk mengerjakan dokumen HTML yang menyediakan API yang baik untuk mengekstraksi, memanipulasi data, dan menyelesaikan pembersihan data awal menggunakan metode terbaik dari \textit{Document Object Model} (DOM), \textit{Cascading Style Sheets} (CSS), dan metode lain yang mirip dengan jQuery. JSOUP mengimplementasikan spesifikasi WHATWG HTML5, dan mem-parsing HTML ke DOM yang sama seperti yang dilakukan \textit{browser} modern. Berikut adalah layanan utama yang tersedia di JSOUP \cite{cokrowibowo}:
\begin{enumerate}
    \item \textit{Scrape} dan \textit{parse} HTML dari URL, \textit{file}, atau string.
    \item Mencari dan ekstrak data menggunakan traversal DOM dan CSS \textit{selector}.
    \item Memanipulasi elemen HTML, atribut HTML, dan teks.
    \item Membersihkan konten yang dikirim oleh pengguna yang menggunakan \textit{safe white-lists} untuk mencegah serangan XSS.
    \item Menghasilkan HTML yang rapi.
\end{enumerate}

Subbab-subbab berikut menjelaskan beberapa kelas dari JSOUP.

\subsection{JSOUP}
Kelas ini merupakan inti untuk mengakses fungsi jsoup. Salah satu fungsi yang dimiliki kelas
ini adalah sebagai berikut:

\begin{itemize}
    \item \textbf{public static Connection connect(String url)}\\
    Berfungsi untuk membuat koneksi baru dengan suatu situs web.\\
    \textbf{Parameter:} URL situs web dengan protokol HTTP atau HTTPS.\\
    \textbf{Kembalian:} koneksi dengan situs web.
\end{itemize}

\subsection{Connection}
Kelas ini merupakan \textit{interface} yang menyediakan pengambilan data dari situs web. Beberapa
fungsi yang dimiliki kelas ini adalah sebagai berikut:

\begin{itemize}
	\item \textbf{Connection cookies(Map<String,String> cookies)} \\
		Berfungsi untuk menambahkan \textit{cookie}. \\
		\textbf{Parameter:}
		\begin{itemize}
			\item \textbf{cookies} \texttt{Map} dari \textit{cookie}.
		\end{itemize}
		\textbf{Kembalian:} koneksi yang sama tetapi sudah diubah.
		
		\item \textbf{Connection data(String key, String value)} \\
		Berfungsi untuk menambahkan parameter data yang bisa dikirim melalui metode HTTP GET atau POST. \\
		\textbf{Parameter:}
		\begin{itemize}
			\item \textbf{key} kunci data.
			\item \textbf{value} nilai data.
		\end{itemize}
		\textbf{Kembalian:} koneksi yang sama tetapi sudah diubah.
		
		\item \textbf{Connection method(Connection.Method method)} \\
		Berfungsi untuk mengatur metode permintaan HTTP, GET atau POST. Metode pengiriman secara \textit{default} adalah GET.\\
		\textbf{Parameter:}
		\begin{itemize}
			\item \textbf{method} metode pengiriman permintaan HTTP.
		\end{itemize}
		\textbf{Kembalian:} koneksi yang sama tetapi sudah diubah.
		
		\item \textbf{Connection timeout(int millis)} \\
		Berfungsi untuk mengatur batas waktu \textit{request}. Batas waktu nol akan dianggap sebagai batas waktu yang tak terhingga. \\
		\textbf{Parameter:}
		\begin{itemize}
			\item \textbf{millis} batas waktu dalam milidetik.
		\end{itemize}
		\textbf{Kembalian:} koneksi yang sama tetapi sudah diubah.
		
		\item \textbf{Connection validateTLSCertificates(boolean value)} \\
		Berfungsi untuk mengatur pemeriksaan sertifikat TLS untuk permintaan HTTPS. Nilai \texttt{true} untuk memeriksa dan nilai \texttt{false} untuk tidak memeriksa.\\
		\textbf{Parameter:}
		\begin{itemize}
			\item \textbf{value} status pemeriksaan sertifikat TLS.
		\end{itemize}
		\textbf{Kembalian:} koneksi yang sama tetapi sudah diubah.
		
		\item \textbf{Connection.Response execute()} \\
		Berfungsi untuk mengirim permintaan HTTP.\\
		\textbf{Kembalian:} objek \texttt{Response}.	
\end{itemize}

\subsection{Response}

Kelas ini merepresentasikan permintaan HTTP. Beberapa \textit{method} yang dimiliki kelas ini adalah sebagai berikut:
\begin{itemize}
	\item \textbf{Map<String,String> cookies()} \\
		\textit{Method} ini berfungsi untuk mendapatkan seluruh \textit{cookies}. \\
		\textbf{Kembalian:} seluruh \textit{cookies}.	
		
		\item \textbf{Document parse()} \\
		Berfungsi untuk mengurai \textit{body} jawaban menjadi dokumen. \\
		\textbf{Kembalian:} koneksi yang sama tetapi sudah diubah.
		
		\item \textbf{String body()} \\
		Berfungsi untuk mendapatkan \textit{body} jawaban dalam bentuk \textit{string}. \\
		\textbf{Kembalian:} \textit{body} jawaban dalam bentuk \textit{string}.
\end{itemize}

\subsection{Document}

Kelas ini merepresentasikan dokumen HTML. Salah satu \textit{method} yang dimiliki kelas ini adalah sebagai berikut:
\begin{itemize}
	\item \textbf{public Elements select(String cssQuery)} \\
		\textit{Method} ini diturunkan dari kelas Element, berfungsi untuk menemukan elemen HTML yang sesuai dengan kueri CSS. \\
		\textbf{Parameter:} 
		\begin{itemize}
			\item \textbf{cssQuery} kueri CSS berupa CSS Selector.
		\end{itemize}
		\textbf{Kembalian:} elemen-elemen HTML yang sesuai dengan kueri CSS.	
\end{itemize}

\subsection{Elements}

Kelas ini merepresentasikan kumpulan elemen HTML. Beberapa \textit{method} yang dimiliki kelas ini adalah sebagai berikut:
\begin{itemize}
	\item \textbf{public Elements select(String query)} \\
		Berfungsi untuk menemukan elemen-elemen yang sesuai dalam \textit{list} elemen. \\
		\textbf{Parameter:} 
		\begin{itemize}
			\item \textbf{query} kueri CSS berupa CSS Selector.
		\end{itemize}
		\textbf{Kembalian:} elemen-elemen yang sudah diseleksi sesuai kueri.	
		
		\item \textbf{public String val()} \\
		Berfungsi untuk mendapatkan nilai dari elemen pertama. \\
		\textbf{Kembalian:} nilai elemen.	
		
		\item \textbf{public String text()} \\
		\textit{Method} Berfungsi untuk mendapatkan kombinasi teks dari seluruh elemen yang sesuai. \\
		\textbf{Kembalian:} seluruh teks dalam \textit{string}.	
\end{itemize}

\subsection{Element}

Kelas ini merepresentasikan sebuah elemen HTML yang berisikan \textit{tag}, atribut, dan anak elemen. Beberapa \textit{method} yang dimiliki kelas ini adalah sebagai berikut:
\begin{itemize}
	\item \textbf{public Element child(int index)} \\
		Berfungsi untuk mendapatkan anak elemen berdasarkan nomor indeks. \\
		\textbf{Parameter:} 
		\begin{itemize}
			\item \textbf{index} nomor index.
		\end{itemize}
		\textbf{Kembalian:} anak elemen.	
		
		\item \textbf{public Element children()} \\
		Berfungsi untuk mendapatkan seluruh anak elemen. \\
		\textbf{Kembalian:} seluruh anak elemen.	
		
		\item \textbf{public String className()} \\
		Berfungsi untuk mendapatkan nama kelas elemen. \\
		\textbf{Kembalian:} nama kelas elemen.	
		
		\item \textbf{public String text()} \\
		Berfungsi untuk mendapatkan teks dari elemen. \\
		\textbf{Kembalian:} teks dalam \textit{string}.	
\end{itemize}

\section{Java Swing}
\label{sec:javaswing}

Java Swing adalah toolkit \textit{graphical user interface} (GUI) generasi berikutnya yang dibuat oleh Sun Microsystems untuk memungkinkan pengembangan perusahaan di Java \cite{loy}. Pada Java, istilah \textit{frame} digunakan untuk apa yang umumnya disebut '\textit{window}'. Pada Swing, \textit{frame} tersebut direalisasikan oleh kelas JFrame. \textit{Frame} adalah area persegi panjang dengan \textit{title bar} di atasnya. \textit{Title bar} juga berisi tombol untuk menutup, memaksimalkan atau membuat ikon bingkai. Di bawah \textit{title bar} adalah area di mana komponen grafis lebih lanjut dapat ditambahkan \cite{fischer}. JFrame tersebut akan dimanfaatkan untuk membuat tampilan dari aplikasi \textit{screen saver}. 


