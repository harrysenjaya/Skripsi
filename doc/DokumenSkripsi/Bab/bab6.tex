\chapter{Kesimpulan dan Saran}
\label{chap:kesimpulan_saran}

\section{Kesimpulan}
\label{sec:kesimpulan}
Dari hasil pembangunan aplikasi \textit{screensaver}, didapatkan kesimpulan-kesimpulan sebagai berikut:
\begin{enumerate}
	\item Aplikasi \textit{screensaver} telah dapat mengambil data-data yang dibutuhkan untuk ditampilkan dengan baik. Namun \textit{screensaver} dengan metode \textit{scraping} ini memiliki kelemahan yaitu jika terdapat perubahan struktur dan penyedia layanan data berhenti atau menghilangkan data yang dibutuhkan, maka data tersebut tidak dapat ditampilkan.
	\item Aplikasi \textit{screensaver} telah dapat dijalankan pada seluruh perangkat dengan sistem operasi windows.
	
\end{enumerate}

\section{Saran}
\label{sec:saran}
Dari hasil penelitian termasuk kesimpulan yang didapat, berikut adalah beberapa saran untuk pengembangan lebih lanjut:
\begin{enumerate}
    \item Aplikasi \textit{screensaver} sebaiknya mengganti metode pengambilan data yang sebelumnya menggunakan metode \textit{scraping} diganti dengan metode lain, sehingga jika terjadi perubahan struktur, atau penyedia layanan data berhenti, atau menghilangkan data yang dibutuhkan, maka masih dapat menampilkan data yang sesuai dengan kebutuhan aplikasi.
    \item Tampilan aplikasi \textit{screensaver} sebaiknya dibuat secara \textit{responsive}, sehingga aplikasi \textit{screensaver} dapat dijalankan dengan baik di berbagai resolusi layar.
    \item Pengambilan data mahasiswa pada aplikasi \textit{screensaver} sebaiknya dijalankan secara paralel, sehingga ketika tampilan berubah ke mahasiswa selanjutnya tidak perlu menunggu pengambilan data mahasiswa tersebut terlebih dahulu.
    
\end{enumerate}